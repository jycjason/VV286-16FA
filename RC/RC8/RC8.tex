\documentclass{beamer}
\usepackage{amsfonts}
\usepackage{amsmath}
\usepackage{times}
\usepackage{mathrsfs}
\usepackage{extarrows}
\usepackage{bbm} 
\usepackage{ulem}
\setbeamercolor{footcolor}{fg=blue!100} % 设置字体和背景颜色
\setbeamertemplate{headline}{%
  \leavevmode%
  \hbox{%
    \hskip228pt
    \begin{beamercolorbox}[wd=.126\paperwidth,ht=2.25ex,dp=1ex,right]{footcolor}%      
       \textcolor[rgb]{0,0.168,0.376}{Slide \insertframenumber{} }
    \end{beamercolorbox}}%
  \vskip-19pt%
}

\setbeamertemplate{frametitle}
{
\vspace{30pt}\textcolor[rgb]{0,0.168,0.376}{\insertframetitle}
}

 
\pgfdeclareimage[height=0.61cm]{university-logo}{logo.png}  
\logo{\pgfuseimage{university-logo}{\vspace{244pt}}} 
\title{\textcolor[rgb]{0,0.168,0.376}{VV286 RC8}}
\author{JIANG Yicheng}
\begin{document}

\begin{frame}
\titlepage
\end{frame}

\begin{frame}
\frametitle{Fourier Series}
The space $L^2([a,b])$ is defined as the completion of $C([a, b])$ with respect to the norm $||\cdot||_2$
$$||f||_2=\sqrt{\int_a^b|f(x)|^2dx}$$
$$L^2([a,b]):=\Big\lbrace f:[a,b]\rightarrow\mathbb{C}:\int_a^b|f(x)|^2dx<\infty\Big\rbrace$$
with inner product
$$\langle f,g\rangle:=\int_a^b\overline{f(x)}g(x)dx\hspace{6mm}f,g\in L^2([a,b])$$

\end{frame}

\begin{frame}
The orthonormal system basis in $L^2([-1,1])$
$$\mathcal{B}_{[-1,1]}=\Big\lbrace \dfrac{1}{\sqrt{2}},\cos(\pi nx),\sin(\pi nx)\Big\rbrace_{n=1}^{\infty}$$
The orthonormal system basis in $L^2([a,b])$
$$\mathcal{B}_{[a,b]}=\Big\lbrace \tilde{e_n}(x):\tilde{e_n}(x)=\sqrt{\dfrac{2}{b-a}}\cdot e_n\Big(\dfrac{2}{b-a}\Big(x-\dfrac{b+a}{2}\Big)\Big)\Big\rbrace_{n=1}^{\infty}$$
where $e_n(x)\in\mathcal{B}_{[-1,1]}$
\end{frame}

\begin{frame}
\begin{block}{The basis in $L^2([0,L])$}
\begin{enumerate}
\item The Fourier-Euler Basis:
$$\mathcal{B}_1=\Big\lbrace \dfrac{1}{\sqrt{L}},\sqrt{\dfrac{2}{L}}\cos\Big(\dfrac{2\pi nx}{L}\Big),\sqrt{\dfrac{2}{L}}\sin\Big(\dfrac{2\pi nx}{L}\Big)\Big\rbrace_{n=1}^{\infty}$$
\item The Fourier-Cosine Basis:
$$\mathcal{B}_2=\Big\lbrace \dfrac{1}{\sqrt{L}},\sqrt{\dfrac{2}{L}}\cos\Big(\dfrac{\pi nx}{L}\Big)\Big\rbrace_{n=1}^{\infty}$$
\item The Fourier-Sine Basis:
$$\mathcal{B}_3=\Big\lbrace \sqrt{\dfrac{2}{L}}\sin\Big(\dfrac{\pi nx}{L}\Big)\Big\rbrace_{n=1}^{\infty}$$
\end{enumerate}
\end{block}
\end{frame}

\begin{frame}

\begin{block}{The basis in $L^2([-\pi,\pi])$}
The Fourier-Euler Basis:
$$\mathcal{B}_1=\Big\lbrace \dfrac{1}{\sqrt{2\pi}},\dfrac{1}{\sqrt{\pi}}\cos(nx),\dfrac{1}{\sqrt{\pi}}\sin(nx)\Big\rbrace_{n=1}^{\infty}$$

\begin{block}{Complex Fourier-Euler Basis}
Complex Fourier-Euler basis in $L^2([-\pi,\pi])$
$$\mathcal{B}=\Big\lbrace \dfrac{1}{\sqrt{2\pi}}e^{inx}\Big\rbrace_{n=-\infty}^{\infty}$$
\end{block}
Complex Fourier-Euler basis in $L^2([-L,L])$
$$\mathcal{B}=\Big\lbrace \dfrac{1}{\sqrt{2L}}e^{inx\pi/L}\Big\rbrace_{n=-\infty}^{\infty}$$
\end{block}
\end{frame}
\begin{frame}
\begin{block}{The Fourier Expansion}
For a function $f\in L^2([a,b])$, 
\begin{align*}
\lim\limits_{N\rightarrow\infty}S_N(x)&=\sum\limits_{e_n\in\mathcal{B}_{[a,b]}}\langle e_n(x),f(x) \rangle e_n(x)\\
&=\sum\limits_{e_n\in\mathcal{B}_{[a,b]}}\int_{a}^{b} f(x)\overline{e_n(x)} dx\cdot e_n(x)
\end{align*}
\end{block}
\end{frame}
\begin{frame}
\begin{block}{Complex Fourier series for periodic function}
$$f(x)=\sum\limits_{k=-\infty}^{\infty}a_ke^{jk\frac{2\pi}{T}x}$$
$$a_k=\dfrac{1}{T}\int_{0}^{T}f(x)e^{-jk\frac{2\pi}{T}x}dx$$
\end{block}
\end{frame}




\begin{frame}
\begin{block}{Dirichlet's rule}
Let $f \in L^2([a, b])$ be piecewise continuously differentiable. Then
\begin{enumerate}
\item On any subinterval $[a', b'] \subset [a, b]$ with $a' > a, b' < b$ on which $f$ is continuous the Fourier series converges uniformly towards $f$ .
\item At any point $x \in [a, b]$, we have the pointwise limit
$$S_N(x)\xrightarrow{N\rightarrow\infty}\dfrac{\lim\limits_{y\nearrow x}f(y)+\lim\limits_{y\searrow x}f(y)}{2}$$
\end{enumerate}
\end{block}
\end{frame}




\begin{frame}
\begin{block}{Using Fourier series to evaluate series}

\end{block}
\begin{block}{Example}
Let $f:\mathbb{R}\rightarrow\mathbb{R}$ satisfy $f(x+2\pi)=f(x)$ and
$$f(x)=e^x\hspace{6mm}\text{for }-\pi<x<\pi$$
Find the Fourier series of $f$ and use it to evaluate
$$\sum\limits_{n=1}^{\infty}\dfrac{1}{1+n^2}$$
\end{block}
\end{frame}
\begin{frame}
\begin{block}{Solution}
\begin{align*}
&\int_{-\pi}^{\pi}f(x)\cdot\dfrac{1}{\sqrt{2\pi}}e^{-inx}\,\,\, dx=\int_{-\pi}^{\pi}e^x\cdot\dfrac{1}{\sqrt{2\pi}}e^{-inx}\,\,\, dx\\
=&\dfrac{1}{\sqrt{2\pi}}\dfrac{1}{1-ni}e^xe^{-inx}\Big|_{-\pi}^{\pi}\\
=&\dfrac{1}{\sqrt{2\pi}}\dfrac{(-1)^n}{1-ni}(e^{\pi}-e^{-\pi})
\end{align*}
Let $x=\pi$, then
\begin{align*}
&\sum\limits_{n=-\infty}^{\infty}\dfrac{1}{\sqrt{2\pi}}\dfrac{(-1)^n}{1-ni}(e^{\pi}-e^{-\pi})\cdot \dfrac{1}{\sqrt{2\pi}}e^{inx}\Big|_{x=\pi}\\
=&\dfrac{e^{\pi}-e^{-\pi}}{2\pi}\sum\limits_{n=-\infty}^{\infty}\dfrac{1}{1-ni}=\dfrac{e^{\pi}-e^{-\pi}}{2\pi}\Big(1+2\sum\limits_{n=1}^{\infty}\dfrac{1}{1+n^2}\Big)
\end{align*}
\end{block}
\end{frame}
\begin{frame}
\begin{block}{}
According to Dirichlet's rule,
\begin{align*}
\dfrac{e^{\pi}-e^{-\pi}}{2\pi}\Big(1+2\sum\limits_{n=1}^{\infty}\dfrac{1}{1+n^2}\Big)&=\dfrac{\lim\limits_{y\nearrow \pi}f(y)+\lim\limits_{y\searrow \pi}f(y)}{2}\\
&=\dfrac{e^{\pi}+e^{-\pi}}{2}
\end{align*}
So
\begin{align*}
\sum\limits_{n=1}^{\infty}\dfrac{1}{1+n^2}=\dfrac{1}{2}\Big(\dfrac{\pi(e^{\pi}+e^{-\pi})}{e^{\pi}-e^{-\pi}}-1\Big)
\end{align*}
\end{block}
\end{frame}

\begin{frame}
\frametitle{Partial Differential Equation}
\begin{enumerate}
\item Use separation of variables
$$u(x_1,\cdots, x_n) = u_1(x_1) \cdot u_2(x_2) \cdots u_n(x_n)$$
\item Change the boundary condition
\item Solve one equation $L u=\lambda u$ with initial consition and boundary condition (find eigenvalues together with eigenfunction)
\item Solve other equations
\item Let the whole solution satisfy boundary condition 
\begin{enumerate}
\item Usually expand boundary conditions to Fourier series and determine the coefficients
\end{enumerate}
\end{enumerate}

\end{frame}
\begin{frame}
\begin{block}{Example}
Find the general solution of the Laplace equation $\triangle u = 0$ on the rectangle $\Omega = [0,a] \times [0,b]$ with boundary conditions
$$u(0, y) = 0, u(a,y) = 0, 0 < y < b$$
$$\dfrac{\partial u}{\partial y}(x,0) = 0, u(x, b) = g(x), 0 < x < a$$
where $g : [0, a] \rightarrow \mathbb{R}$ is a continuous function. Then find the solution if
$$g(x) =\left\{
\begin{aligned}
&x,0<x<a/2\\
&a-x,a/2<x<a\\
\end{aligned}
\right.
$$
\end{block}
\end{frame}
\begin{frame}
\begin{block}{Solution}
We make a separation-of-variables ansatz $u(x, y) = X(x)Y (y)$, yielding
$$\dfrac{\partial^2 u}{\partial x^2}+\dfrac{\partial^2 u}{\partial y^2}=0\Rightarrow X''Y+XY''=0$$
and hence
$$\dfrac{X''}{X}=-\dfrac{Y''}{Y}=\lambda$$
So
$$X''=\lambda X, Y''=-\lambda Y $$
with initial conditions and boundary conditions
$$X(0)Y(y)=0,X(a)Y(y)=0, 0<y<b$$
$$X(x)Y'(0)=0,X(x)Y(b)=g(x),0<x<a$$
\end{block}
\end{frame}
\begin{frame}
To find non-trival solution, the condition can be changed to
$$X(0)=X(a)=Y'(0)=0,X(x)Y(b)=g(x),0<x<a$$
For $X''=\lambda X$, we use ansatz $X(x)=e^{\rho(\lambda)x}$, and therefore
$$\rho(\lambda)^2=\lambda$$
\begin{block}{1. $\lambda=0$}
$X(x)=c_0+c_1x$ to satisfy the initial conditions and boundary conditions, $X(x)=0$
\end{block}
\end{frame}
\begin{frame}
\begin{block}{2. $\lambda>0$}
\begin{align*}
\rho(\lambda)=\pm\sqrt{\lambda}\Rightarrow X(x)=c_1e^{\sqrt{\lambda}x}+c_2e^{-\sqrt{\lambda}x}
\end{align*}
Since $X(0)=X(a)=0$, 
$$c_1+c_2=0,c_1e^{\sqrt{\lambda}a}+c_2e^{-\sqrt{\lambda}a}=0\Rightarrow c_1=c_2=0$$
\end{block}

\end{frame}
\begin{frame}
\begin{block}{3. $\lambda<0$}
\begin{align*}
\rho(\lambda)=\pm\sqrt{-\lambda}i\Rightarrow X(x)&=c_1e^{i\sqrt{-\lambda}x}+c_2e^{-i\sqrt{-\lambda}x}\\
&=c_1\sin(\sqrt{-\lambda}x)+c_2\cos(\sqrt{-\lambda}x)
\end{align*}
Since $X(0)=X(a)=0$, 
$$c_2=0,c_1\sin(\sqrt{-\lambda}a)+c_2\cos(\sqrt{-\lambda}a)=0\Rightarrow \sqrt{-\lambda}=\dfrac{n\pi}{a}$$
So 
$$X_n(x)=c_n\sin(\dfrac{n\pi x}{a}),Y''=\dfrac{n^2\pi^2}{a^2}Y$$
Then
$$Y_n(y)=d_ne^{n\pi y/a}+f_ne^{-n\pi y/a}$$
\end{block}
\end{frame}
\begin{frame}
Considering $Y'(0)=0$
$$d_n\cdot n\pi/a-f_n\cdot n\pi/a=0\Rightarrow d_n=f_n$$
So
$$u(x,y)=\sum\limits_{n=1}^{\infty}X_nY_n=\sum\limits_{n=1}^{\infty}c_nd_n(e^{n\pi y/a}+e^{-n\pi y/a})\sin(\dfrac{n\pi x}{a})$$
Expand $g(x)$ into Fourier sine series
\begin{align*}
&\int_0^ag(x)\sin (n\pi x/a)dx\\
=&\int_0^{a/2}x\sin(n\pi x/a)dx+\int_{a/2}^a(a-x)\sin(n\pi x/a)dx\\
=&\dfrac{2a^2}{n^2\pi^2}\sin(n\pi/2)
\end{align*}
\end{frame}
\begin{frame}
So 
$$g(x)=\sum\limits_{n=1}^{\infty}\dfrac{2}{a}\Bigg(\dfrac{2a^2}{n^2\pi^2}\sin(n\pi/2)\Bigg)\sin(\dfrac{n\pi x}{a})$$
Since $u(x,b)=g(x)$
\begin{align*}
&c_nd_n(e^{n\pi b/a}+e^{-n\pi b/a})=\dfrac{4a}{n^2\pi^2}\sin(n\pi/2)
\end{align*}
So $$u(x,y)=\sum\limits_{n=1}^{\infty}\dfrac{4a\sin(n\pi/2)}{e^{n\pi b/a}+e^{-n\pi b/a}}(e^{n\pi y/a}+e^{-n\pi y/a})\sin(\dfrac{n\pi x}{a})$$
\end{frame}

\begin{frame}
\frametitle{Summary}
\begin{itemize}
\item Use ansatz to find power series solution of non-constant coefficient 2nd-order ODE 
\begin{enumerate}
\item Normal one: $x(t)=\sum\limits_{n=0}^{\infty}a_nt^n$
\item Frobenius: $x(t)=t^r\sum\limits_{n=0}^{\infty}a_nt^n$
\begin{enumerate}
\item $r_1,r_2\in\mathbb{R}$, $r_1-r_2\notin\mathbb{N}$
\item $r_1,r_2\in\mathbb{R}$, $r_1-r_2\in\mathbb{N}$: 

1)try; 

2)$\dfrac{dx_1(t)}{dr}\Big|_{r=r_2}\Rightarrow a'_n(r)$

\item $r_1,r_2\in\mathbb{C}$
\end{enumerate}
\end{enumerate}
\end{itemize}
\end{frame}

\begin{frame}
\begin{itemize}
\item Change equation to Bessel Equation and find solution
\begin{enumerate}
\item Given \sout{(or find)} substitution (see \textit{bessel.pdf})
$$z=g(x)\xlongequal{\text{usually}}\beta x^{\gamma}, u(g(x))=v(x,y(x))\xlongequal{\text{usually}}x^{\alpha}y(x)$$
\item If no information of order, calaulate
\begin{align*}
y'(x)&=-\alpha x^{-\alpha-1}u(g(x))+x^{-\alpha}\dfrac{du(z)}{dz}\Bigg|_{z=g(x)}\dfrac{d(g(x))}{dx}\\
&=-\alpha x^{-1}y(x)+x^{-\alpha}u'(z)|_{z=g(x)}g'(x)
\end{align*}
$$y''(x)=\dfrac{d}{dx}(-\alpha x^{-1}y(x)+x^{-\alpha}u'(z)|_{z=g(x)}g'(x))$$
\item Insert and find solution of corresponding Bessel equation $J_{\nu}(z),J_{-\nu}(z)$, then the soltuion is $$y(x)=c_1x^{-\alpha}J_{\nu}(g(x))+c_2x^{-\alpha}J_{-\nu}(g(x))$$
\end{enumerate}

\end{itemize}
\end{frame}

\begin{frame}
\begin{itemize}
\item Evaluate series using Fouries series
\begin{enumerate}
\item According to domain of given function, use proper basis for Fourier series
\begin{enumerate}
\item $[0,L]$: $\mathcal{B}=\Big\lbrace \dfrac{1}{\sqrt{L}},\sqrt{\dfrac{2}{L}}\cos\Big(\dfrac{2\pi nx}{L}\Big),\sqrt{\dfrac{2}{L}}\sin\Big(\dfrac{2\pi nx}{L}\Big)\Big\rbrace_{n=1}^{\infty}$
\item $[-\pi,\pi]$: $\mathcal{B}=\Big\lbrace \dfrac{1}{\sqrt{2\pi}},\dfrac{1}{\sqrt{\pi}}\cos(nx),\dfrac{1}{\sqrt{\pi}}\sin(nx)\Big\rbrace_{n=1}^{\infty}$
\item $[-L,L]$: $\mathcal{B}=\Big\lbrace \dfrac{1}{\sqrt{2L}}e^{inx\pi/L}\Big\rbrace_{n=-\infty}^{\infty}$
\end{enumerate}
\item Calculate the coefficients (Pay attention to constant!)
\item Choose appropriate value for $x$ and use \underline{\textbf{\textit{\emph{Dirichlet's rule}}}} 
\end{enumerate}
\end{itemize}
\end{frame}

\begin{frame}
\begin{itemize}
\item Solve PDE (usually only two variables)

\begin{enumerate}
\item Use separation of variables
$$u(x_1,\cdots, x_m) = u_1(x_1) \cdot u_2(x_2) \cdots u_m(x_m)$$

\item Change the boundary condition
\item  According to boundary condition, choose one equation $L u_1=\lambda u_1$ to solve. (find proper $\lambda$)
\item Solve other equations for each $\lambda$
$$(u_i)_n(x_i)=(c_i)_n\cdot f(\rho(\lambda)_nx_i)$$
\item Let the whole solution satisfy boundary condition 
$$u(x_1,\cdots,x_m)=\sum\limits_{n=1}^{\infty}\Big(\prod\limits_{i=1}^m(u_i)_n(x_i)\Big)$$
\begin{enumerate}
\item Usually expand boundary conditions to Fourier series and determine the coefficients
\end{enumerate}
\end{enumerate}
\end{itemize}
\end{frame}

\end{document}
